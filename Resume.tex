% !TEX program = xelatex
\documentclass[10pt,A4]{article}

% provides \isempty test
\usepackage{xstring, xifthen}
\usepackage{fontspec}
\usepackage{xeCJK}
\setmainfont{Times New Roman}
\setCJKmainfont{Noto Serif CJK SC}

% set font default
\renewcommand*\familydefault{\sfdefault} 	
\usepackage[T1]{fontenc}

\usepackage{enumitem}
\setlist[itemize]{topsep=0pt,itemsep=1pt,leftmargin=*}
% more font size definitions
\usepackage{moresize}

%----------------------------------------------------------------------------------------
%	FONT AWESOME ICONS
%---------------------------------------------------------------------------------------- 

% include the fontawesome icon set
\usepackage{fontawesome}

% use to vertically center content
% credits to: http://tex.stackexchange.com/questions/7219/how-to-vertically-center-two-images-next-to-each-other
\newcommand{\vcenteredinclude}[1]{\begingroup
\setbox0=\hbox{\includegraphics{#1}}%
\parbox{\wd0}{\box0}\endgroup}

% use to vertically center content
% credits to: http://tex.stackexchange.com/questions/7219/how-to-vertically-center-two-images-next-to-each-other
\newcommand*{\vcenteredhbox}[1]{\begingroup
\setbox0=\hbox{#1}\parbox{\wd0}{\box0}\endgroup}

% icon shortcut
\newcommand{\icon}[3] { 							
	\makebox(#2, #2){\textcolor{maincol}{\csname fa#1\endcsname}}
}	

% icon with text shortcut
\newcommand{\icontext}[4]{ 						
	\vcenteredhbox{\icon{#1}{#2}{#3}}  \hspace{2pt}  \parbox{0.9\mpwidth}{\textcolor{#4}{#3}}
}

% icon with website url
\newcommand{\iconhref}[5]{ 						
    \vcenteredhbox{\icon{#1}{#2}{#5}}  \hspace{2pt} \href{#4}{\textcolor{#5}{#3}}
}

% icon with email link
\newcommand{\iconemail}[5]{ 						
    \vcenteredhbox{\icon{#1}{#2}{#5}}  \hspace{2pt} \href{mailto:#4}{\textcolor{#5}{#3}}
}

%----------------------------------------------------------------------------------------
%	PAGE LAYOUT  DEFINITIONS
%----------------------------------------------------------------------------------------

% page outer frames (debug-only)
% \usepackage{showframe}		

% we use paracol to display breakable two columns
\usepackage{paracol}

% define page styles using geometry
\usepackage[a4paper]{geometry}

% remove all possible margins
\geometry{top=1cm, bottom=1cm, left=1cm, right=1cm}

\usepackage{fancyhdr}
\pagestyle{empty}

% space between header and content
% \setlength{\headheight}{0pt}

% indentation is zero
\setlength{\parindent}{0mm}

%----------------------------------------------------------------------------------------
%	TABLE /ARRAY DEFINITIONS
%---------------------------------------------------------------------------------------- 

% extended aligning of tabular cells
\usepackage{array}

% custom column right-align with fixed width
% use like p{size} but via x{size}
\newcolumntype{x}[1]{%
>{\raggedleft\hspace{0pt}}p{#1}}%


%----------------------------------------------------------------------------------------
%	GRAPHICS DEFINITIONS
%---------------------------------------------------------------------------------------- 

%for header image
\usepackage{graphicx}

% use this for floating figures
% \usepackage{wrapfig}
% \usepackage{float}
% \floatstyle{boxed} 
% \restylefloat{figure}

%for drawing graphics		
\usepackage{tikz}				
\usetikzlibrary{shapes, backgrounds,mindmap, trees}

%----------------------------------------------------------------------------------------
%	Color DEFINITIONS
%---------------------------------------------------------------------------------------- 
\usepackage{transparent}
\usepackage{color}

% primary color
\definecolor{maincol}{RGB}{ 225, 0, 0 }

% accent color, secondary
% \definecolor{accentcol}{RGB}{ 250, 150, 10 }

% dark color
\definecolor{darkcol}{RGB}{ 70, 70, 70 }

% light color
\definecolor{lightcol}{RGB}{245,245,245}


% Package for links, must be the last package used
\usepackage[hidelinks]{hyperref}

% returns minipage width minus two times \fboxsep
% to keep padding included in width calculations
% can also be used for other boxes / environments
\newcommand{\mpwidth}{\linewidth-\fboxsep-\fboxsep}
	


%============================================================================%
%
%	CV COMMANDS
%
%============================================================================%

%----------------------------------------------------------------------------------------
%	 CV LIST
%----------------------------------------------------------------------------------------

% renders a standard latex list but abstracts away the environment definition (begin/end)
\newcommand{\cvlist}[1] {
	\begin{itemize}{#1}\end{itemize}
}

%----------------------------------------------------------------------------------------
%	 CV TEXT
%----------------------------------------------------------------------------------------

% base class to wrap any text based stuff here. Renders like a paragraph.
% Allows complex commands to be passed, too.
% param 1: *any
\newcommand{\cvtext}[1] {
	\begin{tabular*}{1\mpwidth}{p{0.98\mpwidth}}
		\parbox{1\mpwidth}{#1}
	\end{tabular*}
}

%----------------------------------------------------------------------------------------
%	CV SECTION
%----------------------------------------------------------------------------------------

% Renders a a CV section headline with a nice underline in main color.
% param 1: section title
\newcommand{\cvsection}[1] {
	\vspace{14pt}
	\cvtext{
		\textbf{\LARGE{\textcolor{darkcol}{\uppercase{#1}}}}\\[-4pt]
		\textcolor{maincol}{ \rule{0.1\textwidth}{2pt} } \\
	}
}

%----------------------------------------------------------------------------------------
%	META SKILL
%----------------------------------------------------------------------------------------

% Renders a progress-bar to indicate a certain skill in percent.
% param 1: name of the skill / tech / etc.
% param 2: level (for example in years)
% param 3: percent, values range from 0 to 1
\newcommand{\cvskill}[3] {
	\begin{tabular*}{1\mpwidth}{p{0.72\mpwidth}  r}
 		\textcolor{black}{\textbf{#1}} & \textcolor{maincol}{#2}\\
	\end{tabular*}%
	
	\hspace{4pt}
	\begin{tikzpicture}[scale=1,rounded corners=2pt,very thin]
		\fill [lightcol] (0,0) rectangle (1\mpwidth, 0.15);
		\fill [maincol] (0,0) rectangle (#3\mpwidth, 0.15);
  	\end{tikzpicture}%
}


%----------------------------------------------------------------------------------------
%	 CV EVENT
%----------------------------------------------------------------------------------------

% Renders a table and a paragraph (cvtext) wrapped in a parbox (to ensure minimum content
% is glued together when a pagebreak appears).
% Additional Information can be passed in text or list form (or other environments).
% the work you did
% param 1: time-frame i.e. Sep 14 - Jan 15 etc.
% param 2:	 event name (job position etc.)
% param 3: Customer, Employer, Industry
% param 4: Short description
% param 5: work done (optional)
% param 6: technologies include (optional)
% param 7: achievements (optional)
\newcommand{\cvevent}[7] {
	
	% we wrap this part in a parbox, so title and description are not separated on a pagebreak
	% if you need more control on page breaks, remove the parbox
	\parbox{\mpwidth}{
		\begin{tabular*}{1\mpwidth}{p{0.72\mpwidth}  r}
	 		\textcolor{black}{\textbf{#2}} & \colorbox{maincol}{\makebox[0.25\mpwidth]{\textcolor{white}{#1}}} \\
			\textcolor{maincol}{\textbf{#3}} & \\
		\end{tabular*}\\[8pt]
	
		\ifthenelse{\isempty{#4}}{}{
			\cvtext{#4}\\
		}
	}

	\ifthenelse{\isempty{#5}}{}{
		\vspace{9pt}
		{#5}
	}

	\ifthenelse{\isempty{#6}}{}{
		\vspace{9pt}
		\cvtext{\textbf{Technologies include:}}\\
		{#6}
	}

	\ifthenelse{\isempty{#7}}{}{
		\vspace{9pt}
		\cvtext{\textbf{Achievements include:}}\\
		{#7}
	}
	\vspace{14pt}
}

%----------------------------------------------------------------------------------------
%	 CV META EVENT
%----------------------------------------------------------------------------------------

% Renders a CV event on the sidebar
% param 1: title
% param 2: subtitle (optional)
% param 3: customer, employer, etc,. (optional)
% param 4: info text (optional)
\newcommand{\cvmetaevent}[4] {
	\textcolor{maincol} {\cvtext{\textbf{\begin{flushleft}#1\end{flushleft}}}}

	\ifthenelse{\isempty{#2}}{}{
	\textcolor{darkcol} {\cvtext{\textbf{#2}} }
	}

	\ifthenelse{\isempty{#3}}{}{
		\cvtext{{ \textcolor{darkcol} {#3} }}\\
	}

	\cvtext{#4}\\[14pt]
}

%---------------------------------------------------------------------------------------
%	QR CODE
%----------------------------------------------------------------------------------------

% Renders a qrcode image (centered, relative to the parentwidth)
% param 1: percent width, from 0 to 1
\newcommand{\cvqrcode}[1] {
	\begin{center}
		\includegraphics[width=0.63\mpwidth]{QR_WeChat.png}
		\includegraphics[width=0.75\mpwidth]{QR_Bilibili.png}
	\end{center}
}


%============================================================================%
%
%
%
%	DOCUMENT CONTENT
%
%
%
%============================================================================%
\begin{document}
\columnratio{0.31}
\setlength{\columnsep}{2.2em}
\setlength{\columnseprule}{4pt}
\colseprulecolor{lightcol}
\begin{paracol}{2}
\begin{leftcolumn}
%---------------------------------------------------------------------------------------
%	META IMAGE
%----------------------------------------------------------------------------------------
\includegraphics[width=0.7\linewidth]{SMX.jpg}	%trimming relative to image size

%---------------------------------------------------------------------------------------
%	META SKILLS
%----------------------------------------------------------------------------------------

\cvsection{个人信息}\\
\begin{itemize}
	\item 姓名: \textbf{孙敏行}
	\item 性别: \textbf{男}
	\item 籍贯: \textbf{山东省泰安市}
	\item 出生年月: \textbf{1998年11月}
\end{itemize} 


\cvsection{个人概况}\\
\normalsize 近年的科研对象是光电跟踪系统和腿式机器人;
具体领域是时域信号滤波,多传感器信号融合,目标状态估计,运动轨迹预测;
实现方法包括鲁棒估计算法、自适应估计算法、神经网络估计算法。\\

\cvsection{工作技能}
\begin{itemize}
\item C/C++、MATLAB、Python
\item ROS系统、嵌入式开发、英语六级
\end{itemize} 

\vfill\null
\begin{center}
    \cvqrcode{0.7}
    \faWeixin\ 401435318\\
    \faPhone\ +86\,18584806027\\
    \faEnvelope\ 401435318@qq.com\\
	\href{https://github.com/ShineMinxing?tab=repositories}{\textbf{\underline{github.com/ShineMinxing}}}\;\raisebox{-0.1ex}{\faExternalLink}\\
	\href{https://space.bilibili.com/33671525}{\textbf{\underline{space.bilibili.com/33671525}}}\;\raisebox{-0.1ex}{\faExternalLink}
\end{center}


%---------------------------------------------------------------------------------------
%	EDUCATION
%----------------------------------------------------------------------------------------
\newpage

\cvsection{个人荣誉}
\begin{itemize}
    \item 2018全国大学生数学建模竞赛国家级一等奖
    \item 2019全国大学生电子设计大赛省级一等奖
    \item 2020山东省优秀毕业生
    \item 2020青岛大学优秀毕业生
    \item 2020青岛大学优秀毕业论文
    \item 2022中国科学院大学优秀学生干部
    \item 2022中国科学院大学三好学生
\end{itemize} 

\cvsection{其他经历}

\begin{itemize}
    \item 青岛大学新媒体部
    \item 青岛大学文明礼仪宣讲团
    \item 四川省凉山州暑期支教
    \item 中科院光电所新媒体部
    \item 青海省黄南州暑期支教
    \item 西藏省山南市寒假支教
\end{itemize} 

\cvsection{个人爱好}
\begin{itemize}
\item 长跑、游泳、滑冰
\item 厨艺、摄影、剪辑
\end{itemize} 

\vfill\null
\begin{center}
    \cvqrcode{0.7}
    \faWeixin\ 401435318\\
    \faPhone\ +86\,18584806027\\
    \faEnvelope\ 401435318@qq.com\\
	\href{https://github.com/ShineMinxing?tab=repositories}{\textbf{\underline{github.com/ShineMinxing}}}\;\raisebox{-0.1ex}{\faExternalLink}\\
	\href{https://space.bilibili.com/33671525}{\textbf{\underline{space.bilibili.com/33671525}}}\;\raisebox{-0.1ex}{\faExternalLink}
\end{center}


\end{leftcolumn}
\begin{rightcolumn}
%---------------------------------------------------------------------------------------
%	TITLE  HEADER
%----------------------------------------------------------------------------------------
\fcolorbox{white}{darkcol}{\begin{minipage}[c][3.5cm][c]{1\mpwidth}
	\begin {center}
		\HUGE{ \textbf{ \textcolor{white}{ 孙敏行 } } } \\[-24pt]
		\textcolor{white}{ \rule{0.1\textwidth}{1.25pt} } \\[4pt]
		\large{ \textcolor{white} {信号滤波与融合估计算法开发} }
	\end {center}
\end{minipage}} \\[14pt]
\vspace{-12pt}

%---------------------------------------------------------------------------------------
%	PROFILE
%----------------------------------------------------------------------------------------

\cvsection{学历}

\cvtext{\textbf{2016 - 2020\;青岛大学\;自动化专业\hfill 工学学士}}\\

\cvtext{\textbf{2020 - 2026\;中国科学院大学\;信号与信息处理专业}}\\

\cvtext{\textbf{\;\;\;\;\;\;\;\;\;\;\;\;\;\;\;\;\;\;\;光电技术研究所\;光场调控科学技术全国重点实验室\hfill 工学博士}}\\

\cvtext{\textbf{2023 - 2024\;新加坡科学技术研究局\;机器人与自动化实验室\hfill 公派留学}}\\

%---------------------------------------------------------------------------------------
%	WORK EXPERIENCE
%----------------------------------------------------------------------------------------
\cvsection{项目经历}

\noindent
\large\textbf{腿式机器人搭载光电吊舱实现巡逻与目标跟踪} \normalsize \hfill \textit{2025.02--2026.06}\\[4pt]
\textbf{现有成果:}实现腿式机器人在复杂地形巡逻,光电吊舱检测目标并跟踪。\\[-4pt]
\begin{itemize}
    \item 设计纯 C++ 腿式机器人\textbf{运动学里程计}(多传感器融合),支持 ROS1/ROS2 便捷迁移(\href{https://github.com/ShineMinxing/Ros2Go2Estimator}{\textbf{\underline{GitHub\faStar{}106}}}\;\raisebox{-0.1ex}{\faExternalLink})
    \item \textbf{Cartographer、KISS-ICP、Fast-LIO2、Point-LIO、LIO-SAM} 实现3D SLAM
	\item \textbf{ROS2 SLAM Toolbox}、\textbf{Nav2-DWBLocalPlanner} 实现2D SLAM与导航
    \item 集成 \textbf{Vosk、DeepSeek、CosyVoice2} 实现语音交互与指令执行
    \item 光电吊舱、机器人位姿、机器人运动协同,实现无人机/人脸跟踪
    \item 实现AR眼镜\href{https://www.bilibili.com/video/BV1rFJHzhEE9/}{\textbf{\underline{引导机器人运动}}\;\raisebox{-0.1ex}{\faExternalLink}}
\end{itemize}

\vspace{1.5em}

\noindent
\large\textbf{分布式运动平台搭载光电吊舱实现无人机追踪} \normalsize \hfill \textit{2025.07--2026.06}\\[4pt]
\textbf{现有成果:}实现复杂场景下的无人机位姿识别,预测运动意图,提高融合估计精度。\\[-4pt]
\begin{itemize}
    \item ROS2系统采集无人机飞行时的IMU信息,光电吊舱IMU、图像信息
    \item 检测视频中的所有运动目标,获取像素位置和轮廓尺寸,辨识其中的\textbf{无人机轨迹}
    \item 解算无人机在图像中的滚转角、俯仰角,并\textbf{自动标注训练数据}
    \item 无人机数据结合\textbf{DOTAv2.0}数据集,完成\textbf{YOLOv11-obb训练}
    \item 通过无人机尺寸估计距离,通过无人机的方位角、滚转角、俯仰角估计加速度
    \item 多个光电吊舱协同工作,实现\href{https://www.bilibili.com/video/BV13we1zEEED/}{\textbf{\underline{分布式传感器的融合估计}}}\;\raisebox{-0.1ex}{\faExternalLink}
\end{itemize}

\vspace{1.5em}

\noindent
\large\textbf{构建跨平台估计算法库} \normalsize \hfill \textit{2024.09--2026.06}\\[4pt]
\textbf{现有成果:}实现Windows、Linux平台,C、C++、Python、Matlab工程的估计算法快速部署。\\[-4pt]
\begin{itemize}
    \item 将估计算法模块化为外部接口、运动模型、估计方法三层架构
    \item 提供\textbf{C/C++/Python/MATLAB}接口,调用\textbf{C/C++/Matlab/DLL/SO}库
    \item 已构建一致的 \textbf{RNN、LSTM、GRU、TCN、NeuralODE、Transformer} 训练与使用模板,下一步将接入算法库
\end{itemize}

\vspace{1.5em}

\noindent
\large\textbf{无人机目标识别及倾角提取} \normalsize \hfill \textit{2024.09--2025.02}\\[4pt]
\textbf{成果:}初步实现无人机无人机位姿识别,预测运动意图,提高估计精度。\\[-4pt]
\begin{itemize}
    \item 录制无人机飞行视频,使用 OpenCV、PyDub 获取目标的像素位置与噪声分贝值
    \item 根据像素位置序列计算加速度(倾角),制作训练数据集
    \item 基于 \textbf{TensorFlow-Keras-MobileNetV2} 迁移学习获取图像处理模型
    \item 将倾角特征作为新信号源,提升无人机位置预测精度
\end{itemize}

\vspace{1.5em}

\noindent
\large\textbf{腿式机器人里程计设计(新加坡科技局)} \normalsize \hfill \textit{2023.09--2024.09}\\[4pt]
\textbf{成果:}设计融合估计器,补偿SLAM算法在低纹理退化环境下的定位失准,改善步态规划与姿态控制。\\[-4pt]
\begin{itemize}
    \item 分析 IMU、SLAM、关节电机编码器数据,设计易扩展的\textbf{融合估计算法框架}
    \item 利用\textbf{落足点记录}与实时足髋位置构建\href{https://www.bilibili.com/video/BV1UtQfYJExu/}{\textbf{\underline{腿式里程计}}}\;\raisebox{-0.1ex}{\faExternalLink},显著提升状态估计精度
    \item 应用容积卡尔曼估计法消除\textbf{运动学数据噪声}
\end{itemize}

\vspace{1.5em}

\noindent
\large\textbf{双反射镜跟踪平台搭建} \normalsize \hfill \textit{2021.09--2023.09}\\[4pt]
\textbf{成果:}搭建由两个快反镜、激光器、CCD 组成的平台,可同时模拟目标运动和测试控制算法。\\[-4pt]
\begin{itemize}
    \item 完成信号解耦与处理,设计\textbf{零极点对消法}的双闭环控制系统
    \item 初步构建和测试 \textbf{C 语言估计算法库}以补偿信号延迟:卡尔曼估计、扩展卡尔曼估计、2种无迹卡尔曼估计、2种交互式多模型估计、鲁棒估计、结合3种不同自适应因子的鲁棒估计、线性回归参数估计、线性回归参数卡尔曼估计
\end{itemize}

\vspace{1.5em}

\noindent
\large\textbf{光轴稳定扰动抑制平台改进} \normalsize \hfill \textit{2019.10--2021.09}\\[4pt]
\textbf{成果:}改进由三个可运动层构成的平台,设计控制算法、估计算法提高跟踪精度。\\[-4pt]
\begin{itemize}
    \item 在VxWorks系统中设计\textbf{零极点对消法}的三闭环控制系统
    \item 设计和测试多种鲁棒估计算法以补偿信号延迟
\end{itemize}

\vspace{1.5em}


\cvsection{科研成果}

\noindent
\normalsize\textbf{论文:Intention Inference based Interacting Multiple Model Estimator in Photoelectric Tracking} 
\small \hfill \textit{IET Control Theory \& Applications} \textit{一作} \\

% \small 传统的估计方法对被跟踪目标的外在物理特征,即位置、速度、加速度进行估计,在面对机动目标时估计精度不足。该论文提出了一种基于运动意图估计的状态估计方法,通过直接估计目标的运动意图参数,如盘旋飞行目标的圆心位置,实现了更精确的状态估计。\\

\noindent
\normalsize\textbf{论文:Multiple Adaptive Factors based Interacting Multiple Model Estimator}\\
\small \phantom{IET Control Theory \& Applications} \hfill \textit{IET Control Theory \& Applications} \textit{一作} \\

% \small 该论文提出了一种基于多自适应因子的交互式鲁棒状态估计器设计方法,突破了单一自适应因子和传统鲁棒估计算法的限制,可以在复杂的观测噪声激增情况下有效提高估计精度和曲线平滑性,优化估计效果。\\

\noindent
\normalsize\textbf{论文:Robust State Estimation for Uncertain Discrete Linear Systems with Delayed Measurements} 
\small \hfill \textit{Mathematics} \textit{学生一作} \\

% \small 该论文针对离散时间线性系统中随机模型参数不确定性和恒定测量延迟的问题,将卡尔曼滤波器正则化最小二乘框架和状态扩张方法结合起来,设计了一种新型的鲁棒估计方法,在处理测量延迟和模型参数不确定性方面表现出更好性能。\\

\noindent
\normalsize\textbf{论文:A Robust State Estimator With Adaptive Factor} 
\small \hfill \textit{IEEE Access} \textit{一作} \\

% \small 该论文提出了一种用于不确定线性系统的鲁棒状态估计器。它利用卡方检验识别异常值并自适应地调整观测噪声协方差,再通过改进的鲁棒估计算法补偿模型不确定性,实现了误差累积的有效降低,且相较传统算法的优势随观测噪声恶化程度的增加而显著提升。\\


\noindent
\normalsize\textbf{专利:一种基于卡方自适应因子的鲁棒控制方法} \small \hfill \textit{学生一作} \\

% CN202011502653.1
% 2020.12.17
% \small 该发明提出了一种基于卡方自适应因子的鲁棒滤波方法,以改善在观测噪声激增情况下的滤波效果。传统鲁棒滤波方法无法适应观测噪声激增,而本发明通过统计学方法快速估计实际观测噪声,有效提升滤波器的滤波精度和曲线平滑性,且额外计算复杂度少。\\

\noindent
\normalsize\textbf{专利:基于多自适应因子的交互式鲁棒状态估计器} \small \hfill \textit{学生一作} \\

% CN202211618386.3
% 2022.12.15
% \small 该发明提出了一种基于多自适应因子的交互式鲁棒状态估计器设计方法,通过多自适应因子的应用,在复杂的观测噪声激增情况下实现了快速估计和优化估计器性能,提高了估计精度和曲线平滑性。\\

\noindent
\normalsize\textbf{专利:一种基于历史状态的变参数卡尔曼滤波器设计方法} \small \hfill \textit{学生一作} \\

% CN202211502311.9
% 2022.11.28
% \small 该发明提出了一种基于历史状态的变参数卡尔曼滤波器设计方法,通过引入常比例参数、变比例参数、离线线性回归参数、在线定存储量线性回归参数和在线增量法线性回归参数的参数设定方法,实现了对滤波器预测效果的有效改善,同时仅增加较少的额外计算复杂度。\\

\noindent
\normalsize\textbf{专利:一种针对周期运动目标的自适应无迹卡尔曼滤波器} \small \hfill \textit{学生一作} \\

% CN202211553537.1
% 2022.12.06
% \small 该发明提出了一种针对运动方程不完全已知的周期运动目标的自适应无迹卡尔曼滤波器设计方法,以改善在状态估计中对周期运动目标的估计误差和滤波曲线的平滑性,满足更高精度的滤波需求,有效改善了无迹卡尔曼滤波器的滤波效果。\\

\noindent
\normalsize\textbf{专利:一种基于意图推定的交互式多模型状态估计方法} \small \hfill \textit{学生一作} \\

% CN202211553542.2
% 2022.12.06
% \small 该发明提出了一种基于意图推定的交互式多模型状态估计方法,通过估计被跟踪目标的运动意图和关键参数,提出了一种基于意图推定的交互式多模型状态估计方法,突破了传统滤波方法的限制,显著提高估计器的估计精度和曲线平滑性,优化了估计器的效果。\\

\noindent
\normalsize\textbf{专利:一种基于扩张状态的参数自整定卡尔曼滤波器设计方法} \small \hfill \textit{学生一作} \\

% CN202211531483.9
% 2022.12.01
% \small 该发明提出了一种基于扩张状态的参数自整定卡尔曼滤波器设计方法,通过实时估计目标的状态方程参数,使得滤波器能够适用于非线性和变轨迹运动目标的跟踪,突破了传统卡尔曼滤波方法的限制,有效提高了滤波器的估计精度和曲线平滑性。\\



% hotfixes to create fake-space to ensure the whole height is used
\mbox{}
\vfill
\end{rightcolumn}
\end{paracol}

\newpage
\cvsection{工作成果展示}

\vspace{-2pt}
\begin{center}\small \textbf{🎥 以下内容是工程开发部署时的节点性记录(可点击跳转)}\end{center}

\begingroup
\footnotesize
\setlength{\tabcolsep}{6pt}
\renewcommand{\arraystretch}{1.15}
\begin{tabular}{p{0.48\linewidth} p{0.48\linewidth}}

% 1 左 | 2 右
\begin{minipage}[t]{\linewidth}\centering
\href{https://www.bilibili.com/video/BV1UtQfYJExu}{\includegraphics[width=0.8\linewidth]{24足切换建图.jpg}}\\
\href{https://www.bilibili.com/video/BV1UtQfYJExu}{\textbf{纯里程计建图 (站立/四足切换)}}
\end{minipage}
&
\begin{minipage}[t]{\linewidth}\centering
\href{https://www.bilibili.com/video/BV18Q9JYEEdn/}{\includegraphics[width=0.8\linewidth]{室内行走.jpg}}\\
\href{https://www.bilibili.com/video/BV18Q9JYEEdn/}{\textbf{室内行走误差 0.5\%-1\%}}
\end{minipage}
\\[24pt]  

% 3 左 | 4 右
\begin{minipage}[t]{\linewidth}\centering
\href{https://www.bilibili.com/video/BV1VV9ZYZEcH/}{\includegraphics[width=0.8\linewidth]{爬楼梯.jpg}}\\
\href{https://www.bilibili.com/video/BV1VV9ZYZEcH/}{\textbf{爬楼梯高度误差 <\,5\,cm}}
\end{minipage}
&
\begin{minipage}[t]{\linewidth}\centering
\href{https://www.bilibili.com/video/BV1BhRAYDEsV/}{\includegraphics[width=0.8\linewidth]{户外行走.jpg}}\\
\href{https://www.bilibili.com/video/BV1BhRAYDEsV/}{\textbf{户外行走 380\,m 误差 3.3\%}}
\end{minipage}
\\[24pt]  

% 5 左 | 6 右
\begin{minipage}[t]{\linewidth}\centering
\href{https://www.bilibili.com/video/BV1HCQBYUEvk/}{\includegraphics[width=0.8\linewidth]{导航和语音交互.jpg}}\\
\href{https://www.bilibili.com/video/BV1HCQBYUEvk/}{\textbf{语音交互 + 地图导航}}
\end{minipage}
&
\begin{minipage}[t]{\linewidth}\centering
\href{https://www.bilibili.com/video/BV1faG1z3EFF/}{\includegraphics[width=0.8\linewidth]{人脸识别跟踪与光点跟踪.jpg}}\\
\href{https://www.bilibili.com/video/BV1faG1z3EFF/}{\textbf{人脸识别跟踪 + 光点跟踪}}
\end{minipage}
\\[24pt]  

% 7 左 | 8 右
\begin{minipage}[t]{\linewidth}\centering
\href{https://www.bilibili.com/video/BV1pXEdzFECW}{\includegraphics[width=0.8\linewidth]{AR眼镜头部运动跟随.jpg}}\\
\href{https://www.bilibili.com/video/BV1pXEdzFECW}{\textbf{AR 眼镜头部运动跟随}}
\end{minipage}
&
\begin{minipage}[t]{\linewidth}\centering
\href{https://www.bilibili.com/video/BV18v8xzJE4G}{\includegraphics[width=0.8\linewidth]{YOLO无人机识别.jpg}}\\
\href{https://www.bilibili.com/video/BV18v8xzJE4G}{\textbf{YOLO 无人机识别与跟随}}
\end{minipage}
\\[24pt]  

% 9 左 | 10 右
\begin{minipage}[t]{\linewidth}\centering
\href{https://www.bilibili.com/video/BV1fTY7z7E5T}{\includegraphics[width=0.8\linewidth]{多图像源融合估计.jpg}}\\
\href{https://www.bilibili.com/video/BV1fTY7z7E5T}{\textbf{机器狗光电吊舱与固定相机协同}}
\end{minipage}
&
\begin{minipage}[t]{\linewidth}\centering
\href{https://www.bilibili.com/video/BV1kjCqBEEK8}{\includegraphics[width=0.8\linewidth]{SLAM方法.jpg}}\\
\href{https://www.bilibili.com/video/BV1kjCqBEEK8}{\textbf{多种SLAM方法效果对比}}
\end{minipage}
\\

\end{tabular}
\endgroup

\end{document}

